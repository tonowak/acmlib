\documentclass[15pt]{article}

\usepackage[a4paper, margin={1cm,2cm}]{geometry}

\usepackage{polski}
\usepackage[utf8]{inputenc}

\usepackage{hyperref}
\hypersetup{
    colorlinks=true,
}

\usepackage{fancyhdr}
\pagestyle{fancy}
\fancyhf{}
\rhead{XIII LO Szczecin, Tomasz Nowak}
\lhead{Techniki na drugie etapy OI}

\begin{document}

\begin{itemize}
    \item BRUTY
    \begin{itemize}
        \item maski bitowe: przejrzenie wszystkich podciągów \href{http://solve.edu.pl/~sparingi/resources}{video}
        \item generowanie wszystkich ciągów n-elementowych
        \begin{itemize}
            \item generowanie wszystkich permutacji
        \end{itemize}
        \item wyznaczanie ścieżki między dwoma wierzchołkami (graf/drzewo)
        \item skomplikowane backtracki
    \end{itemize}
    
    \item DEBUG \href{https://sim.ugo.si/api/download/file/jySGuWzthXf0VT9sONUAjlcojsnaUM}{artykuł}
    \begin{itemize}
        \item pisanie generatorek
        \begin{itemize}
            \item sprawdzarka w bashu \href{https://sim.ugo.si/api/download/file/W7738Bo24gH14MTXjiImqaiwG4iNy8}{plik zip}
            \item sprawdzarka w C++
            \item generowanie losowych liczb, permutacji, drzew, grafów skierowanych / nieskierowanych / spójnych / bez multikrawędzi
        \end{itemize}
        \item wykrywanie błędów
        \begin{itemize}
            \item flagi kompilacji \href{https://sim.ugo.si/api/download/file/1fhJzfNSSRaV0Yax9HAXpwq8xzcQka}{plik zip}
            \item czas wykonania programu oraz zużyta pamięć \href{https://sim.ugo.si/api/download/file/PurVW2KifPwKWUyH9N7NC2PXAnCYZv}{plik zip}
            \item używanie narzędzia gdb do wykrywania runtime error'ów ($\uparrow$ plik zip)
            \item nagłówki ($\uparrow$ pierwszy zip)
        \end{itemize}
    \end{itemize}
    
    \item GRAFY
    \begin{itemize}
        \item Drzewa \href{http://kompendium.meetit.pl/kurs#graph3}{artykuł}
        \begin{itemize}
            \item jump-pointery
            \begin{itemize}
                \item najniższy wspólny przodek (LCA)
                \item operacje na ścieżkach (np. otrzymanie maximum)
            \end{itemize}
            \item właściwości numeracji wierzchołków w drzewie
            \begin{itemize}
                \item preorder, postorder, podobne \href{http://kompendium.meetit.pl/kurs#graph5}{artykuł}
            \end{itemize}
            \item dynamiki na drzewie
            \item znane algorytmy: szukanie centroidów, średnicy drzewa
        \end{itemize}
        
        \item Graf dwudzielny
        \begin{itemize}
            \item określanie dwudzielności grafu
            \item skojarzenia w grafach dwudzielnych \href{http://solve.edu.pl/~sparingi/resources}{video} \href{http://kompendium.meetit.pl/kurs#graph9}{artykuł}
            \item twierdzenie Halla ($\uparrow$ artykuł)
            \item twierdzenie K\"{o}niga \href{http://solve.edu.pl/~sparingi/resources}{video} \href{http://www.deltami.edu.pl/temat/matematyka/teoria_grafow/2013/10/31/W_grafach_dwudzielnych_jest_latwiej/}{artykuł}
        \end{itemize}
        
        \item Skierowany Graf Acykliczny (DAG)
        \begin{itemize}
            \item sortowanie topologiczne (toposort) \href{http://kompendium.meetit.pl/kurs#graphB}{artykuł}
            \item dynamiki na DAGach ($\uparrow$ artykuł)
            \item podział na warstwy
        \end{itemize}
        
        \item Grafy \href{http://kompendium.meetit.pl/kurs#graph1}{artykuł}
        \begin{itemize}
            \item DFS, BFS \href{http://was.zaa.mimuw.edu.pl/?q=node/31}{video} \href{http://kompendium.meetit.pl/kurs#graph1}{artykuł}
            \item drzewo DFS \href{http://was.zaa.mimuw.edu.pl/?q=node/33}{video}
            \begin{itemize}
                \item funkcja low \& mosty, punkty artykulacji, dwuspójne \href{http://solve.edu.pl/~sparingi/resources}{video} \href{http://was.zaa.mimuw.edu.pl/?q=node/39}{video2} \href{http://kompendium.meetit.pl/kurs#graph7}{artykuł}
            \end{itemize}
            \item Dijkstra \href{http://kompendium.meetit.pl/kurs#graph2}{artykuł}
            \item rozbicie wierzchołkowe
            \item potęgowanie macierzy sąsiedztwa \href{http://was.zaa.mimuw.edu.pl/?q=node/35}{video} \href{http://kompendium.meetit.pl/kurs#dp5}{artykuł}
            \item silne spójne składowe (SCC) \href{http://kompendium.meetit.pl/kurs#graphB}{artykuł}
            \item cykl Eulera \href{http://was.zaa.mimuw.edu.pl/?q=node/31}{video} \href{http://kompendium.meetit.pl/kurs#graph1}{artykuł}
            \item minimalne drzewo rozpinające (MST)
            \begin{itemize}
                \item algorytm Kruskala (mniej ważny: algorytm Prima) \href{http://was.zaa.mimuw.edu.pl/?q=node/39}{video} \href{http://kompendium.meetit.pl/kurs#graph7}{artykuł} 
                \item Prufer Code do brutowania wszystkich możliwych drzew \href{https://cp-algorithms.com/graph/pruefer_code.html}{article}
            \end{itemize}
            \item rozbicie wierzchołkowe 
            \item 2-SAT \href{http://solve.edu.pl/~sparingi/resources}{video}
        \end{itemize}
        
        \item Meduzy \href{http://kompendium.meetit.pl/kurs#graph8}{artykuł}
        
        \item Grafy planarne \href{http://kompendium.meetit.pl/kurs#graphD}{artykuł}
        \begin{itemize}
            \item wzór Eulera ($\uparrow$ artykuł)
            \item zapytania o istnienie ścieżki w grafie planarnym ($\uparrow$ artykuł)
        \end{itemize}
    \end{itemize}
    
    \item STRUKTURY DANYCH
    \begin{itemize}
        \item Trie \href{http://kompendium.meetit.pl/kurs#text4}{artykuł}
        \item Drzewa przedziałowe \href{http://solve.edu.pl/~sparingi/resources}{video} \href{http://kompendium.meetit.pl/kurs#struct2}{artykuł}
        \begin{itemize}
            \item drzewa przedział-przedział \href{http://was.zaa.mimuw.edu.pl/?q=node/8}{video} ($\uparrow$ video) ($\uparrow$ artykuł)
            \item drzewa trwałe, wskaźnikowe
        \end{itemize}
        \item Sqrt \href{http://kompendium.meetit.pl/kurs#struct4}{artykuł}
        \begin{itemize}
            \item sqrt decomposition
            \item split \& rebuild
            \item algorytm mo ($\uparrow$ artykuł)
        \end{itemize}
        \item kolejka monotoniczna \href{http://kompendium.meetit.pl/kurs#struct3}{artykuł}
        \item mniejszy do większego
        \item find \& union \href{http://kompendium.meetit.pl/kurs#struct5}{artykuł} \href{http://solve.edu.pl/~sparingi/resources}{video}
        \item sumy prefixowe \href{http://kompendium.meetit.pl/kurs#basic1}{artykuł}
        \item std::vector jako stos, std::deque jako kolejka, std::set, std::map \href{http://kompendium.meetit.pl/kurs#basic7}{artykuł}
        \item sort
        \begin{itemize}
            \item std::sort, komparator i błędy z komparatorem
            \item sortowanie kubełkowe \href{http://kompendium.meetit.pl/kurs#basic5}{artykuł}
            \item skalowanie \href{http://kompendium.meetit.pl/kurs#basic6}{artykuł}
        \end{itemize}
    \end{itemize}
    
    \item DYNAMIKI \href{http://kompendium.meetit.pl/kurs#meth3}{artykuł}
    \begin{itemize}
        \item problem plecakowy (różne odmiany, sqrt optymalizacja) \href{http://kompendium.meetit.pl/kurs#dp1}{artykuł}
        \item dynamiki kombinatoryczne \href{http://kompendium.meetit.pl/kurs#dp2}{artykuł}
        \item dynamiki przedziałowe \href{http://kompendium.meetit.pl/kurs#dp3}{artykuł}
        \item dynamiki optymalizacyjne ($\uparrow$ artykuł)
        \item dynamiki wykładnicze \href{http://kompendium.meetit.pl/kurs#dp4}{artykuł}
        \item potęgowanie macierzy \href{http://was.zaa.mimuw.edu.pl/?q=node/35}{video} \href{http://kompendium.meetit.pl/kurs#dp5}{artykuł}
    \end{itemize}
    
    \item MATMA
    \begin{itemize}
        \item arytmetyka modulo
        \begin{itemize}
            \item najmniejszy wspólny dzielnik (NWD) \href{http://solve.edu.pl/~sparingi/resources}{video} \href{http://kompendium.meetit.pl/kurs#num1}{artykuł}
            \begin{itemize}
                \item rozszerzony algorytm euklidesa, odwrotność modulo \href{http://solve.edu.pl/~sparingi/resources}{video} \href{http://kompendium.meetit.pl/kurs#num2}{artykuł}
            \end{itemize}
            \item funkcja Phi (totient Eulera) \href{http://solve.edu.pl/~sparingi/resources}{video}
            \begin{itemize}
                \item Małe Twierdzenie Fermat (MTF) ($\uparrow$ video)
            \end{itemize}
            \item Chińskie Twierdzenie o Resztach (CTR) ($\uparrow$ video)
            \item szybkie potęgowanie \href{http://kompendium.meetit.pl/kurs#basic4}{artykuł}
        \end{itemize}
        \item znajdowanie dzielników (sqrt), dzielników pierwszych (sito Eratostenesa/sqrt) \href{http://kompendium.meetit.pl/kurs#num1}{artykuł}
        \item teoria gier \href{http://kompendium.meetit.pl/kurs#games1}{artykuł}
        \begin{itemize}
            \item twierdzenie Sprague-Grundy'ego \href{http://solve.edu.pl/~sparingi/resources}{video} \href{http://was.zaa.mimuw.edu.pl/?q=node/18}{video2} \href{http://kompendium.meetit.pl/kurs#games2}{artykuł}
        \end{itemize}
        \item potęgowanie macierzy \href{http://was.zaa.mimuw.edu.pl/?q=node/35}{video} \href{http://kompendium.meetit.pl/kurs#dp5}{artykuł}
        \item podstawy kombinatoryki
        \begin{itemize}
            \item dwumian Newton'a
            \item zasada włączeń i wyłączeń
        \end{itemize}
        \item rozwiązania probabilistyczne, paradoks urodzeń
        \item suma ciągu harmonicznego, złożoność sita
    \end{itemize}
    
    \item GEOMETRIA \href{http://kompendium.meetit.pl/kurs#geo1}{artykuł} \href{http://informatyka.wroc.pl/node/455?page=0,0}{artykuł2}
    \begin{itemize}
        \item iloczyn wektorowy ($\uparrow$ artykuł)
        \item pole wielokąta \href{http://kompendium.meetit.pl/kurs#geo2}{artykuł}
        \item wielokąty wypukłe ($\uparrow$ artykuł)
        \begin{itemize}
            \item otoczka wypukła \href{http://kompendium.meetit.pl/kurs#geo4}{artykuł}
        \end{itemize}
        \item zamiatanie \href{http://was.zaa.mimuw.edu.pl/?q=node/37}{video} \href{http://kompendium.meetit.pl/kurs#geo5}{artykuł}
        \item sortowanie kątowe \href{http://kompendium.meetit.pl/kurs#geo3}{artykuł}
        \item geometria obliczeniowa \href{http://was.zaa.mimuw.edu.pl/?q=node/15}{video} \href{http://was.zaa.mimuw.edu.pl/?q=node/45}{video2}
        \item najbliższa para punktów \href{http://was.zaa.mimuw.edu.pl/?q=node/37}{video}
        \item najdalsza para punktów
    \end{itemize}
    
    \item ALGORYTMY TEKSTOWE \href{http://kompendium.meetit.pl/kurs#text1}{artykuł}
    \begin{itemize}
        \item liniowe algorytmy tekstowe \href{http://was.zaa.mimuw.edu.pl/?q=node/5}{video} \href{http://kompendium.meetit.pl/kurs#text3}{artykuł}
        \begin{itemize}
            \item tablica pi, algorytm KMP \href{http://solve.edu.pl/~sparingi/resources}{video} ($\uparrow$ video)
            \item algorytm Manachera \href{http://was.zaa.mimuw.edu.pl/?q=node/5}{video}
            \item tablica prefixo-prefixowa ($\uparrow$ video)
            \item szablony ($\uparrow$ video)
            \item okresy ($\uparrow$ video)
        \end{itemize}
        \item hashowanie (posłów, podciągów, przypisanie literkom / liczbom losowe wagi i sprawdzanie sumy)($\uparrow$ video) \href{http://kompendium.meetit.pl/kurs#text2}{artykuł}
        \item drzewo Trie \href{http://kompendium.meetit.pl/kurs#text4}{artykuł}
        \item tablica suffixowa \href{http://solve.edu.pl/~sparingi/resources}{video} \href{http://kompendium.meetit.pl/kurs#text7}{artykuł}
    \end{itemize}
    
    \item OPTYMALIZACJE
    \begin{itemize}
        \item dziel i zwyciężaj \href{http://kompendium.meetit.pl/kurs#meth1}{artykuł}
        \begin{itemize}
            \item bardziej skomplikowane przykłady dziel i zwyciężaj
            \item meet in the middle, czyli wzorcówka dla $n \leq 40$ \href{http://kompendium.meetit.pl/kurs#dp4}{artykuł}
        \end{itemize}
        \item wyszukiwanie binarne \href{http://kompendium.meetit.pl/kurs#basic2}{artykuł}
        \begin{itemize}
            \item wyszukiwanie binarne po wyniku
            \item wyszukiwanie binarne wielu rzeczy na raz / równoległe wyszukiwanie binarne
        \end{itemize}
        \item koszt zamortyzowany
        \begin{itemize}
            \item gąsienica \href{http://kompendium.meetit.pl/kurs#basic3}{artykuł}
            \item zadania na stos
        \end{itemize}
        \item skalowanie \href{http://kompendium.meetit.pl/kurs#basic6}{artykuł}
        \item lider ciągu
        \item odpowiednie struktury danych
        \item bitsety
    \end{itemize}
    
    \newpage
    \item PODEJŚCIA
    \begin{itemize}
        \item dynamik
        \item algorytm zachłanny \href{http://kompendium.meetit.pl/kurs#meth2}{artykuł}
        \item sposób "liczba dobrych obiektów = liczba wszystkich obiektów - liczba złych obiektow"
        \item czy warunek konieczny = warunek wystarczający?
        \item odpowiednie przekształcenie równania; uniezależnienie funkcji od jakiejś zmiennej; zauważenie wypukłości
        \item zastanowić się nad łatwiejszym problemem, bez jakiegoś elementu z treści
        \item sprowadzić problem do innego, łatwiejszego/mniejszego problemu
        \item sprowadzić problem 2D do problemu 1D (zamiatanie; niezależność wyniku dla współrzędnych X od współrzędnych Y)
        \item konstrukcja grafu
        \item określenie struktury grafu
        \item napisanie bruta przed wzorcówką (przynajmniej gwarantowane punkty są, a potem sprawdzaczka)
        \item optymalizacja bruta do wzorcówki
        \item czy można poprawić (może zachłannie) rozwiązanie nieoptymalne?
        \item czy są ciekawe fakty w rozwiązaniach optymalnych (może się do tego przydać brute)
        \item sklepać brute który sprawdza obserwacje, zawsze jeśli potrzebujemy zoptymalizować dp, wypisać wartości na małym przykładzie
        \item sprawdzić czy w zadaniu czegoś jest "mało" (np. czy wynik jest mały, albo jakaś zmienna, może się do tego przydać brute)
        \item odpowiednio "wzbogacić" jakiś algorytm (np. Dijkstrę)
        \item spróbować obliczyć wkład do wyniku jakiegoś elementu
        \item jeżeli miałoby się teraz zrobić to zadanie i to nie jest hardkor, jakie obserwacje by pomogły?
        \item cokolwiek poniżej $10^9$ operacji ma szansę wejść na 100 pkt
        \item coś ciekawego w limitach w zadaniu i subtaskach?
        \item obserwacje, obserwacje, obserwacje!
        \item co można wykonać offline? czy jest coś, czego kolejność nie ma znaczenia?
	    \item co można posortować? czy jest zawsze jakaś pewna optymalna kolejność / czy można ustalić jak ta kolejność wygląda?
        \item narysować dużo swoich własnych przykładów i coś z nich wywnioskować
    	\item skupić się na jakimś specjalnym elemencie, najczęściej najmniejszego/największego (np. spojrzeć na jedynkę w permutacji)
	    \item szacowanie wyniku - czy wynik jest mały? czy można skonstruować algorytm, który zawsze znajdzie górne ograniczenie na wynik?
        \item nie gardzić punktami gdy leżą na ulicy (podzadania)
    \end{itemize}
    
    \item UWAGI, CZĘSTE BŁĘDY
    \begin{itemize}
        \item pamiętaj o pamięci... $10^6$ intów $= 4$ MB 
        \item long longi...
        \item na OIu flaga -fsanitize=address lub -fsanitize=undefined może nie działać, trzeba wtedy skorzystać z kompilatora g++-4.9 zamiast clang++
        \item na OIu jakiekolwiek komparatory do sort'ów wymagają, by argumenty były const
        \item funkcja resize() w vectorach nie czyści (nie resetuje wartości) pozostałych elementów - kłopotliwe jak się ma tablice globalne i wiele testów w jednym pliku weściowym
        \item pamiętaj o trzech magicznych linijkach podczas korzystania z cin, cout
        \item nie można szybko sprawdzać odległości między iteratorami w secie/mapie (w szczególności, set nie potrafi szybko odpowiadać na pytania "którym elementem jest x")
        \item jak w treści jest jakiś dziwny warunek i nie wiadomo co on w treści robi (lub jakaś zmienna jest podejrzanie mała), jest to klucz do rozwiązania
    \end{itemize}
    
    \item MATERIAŁY
    \begin{itemize}
        \item \href{http://solve.edu.pl/~sparingi/resources}{solve.edu.pl/\textasciitilde sparingi/resources}
        \item \href{http://was.zaa.mimuw.edu.pl}{was.zaa.mimuw.edu.pl} (lepsza jakość nagrań na \href{https://www.youtube.com/channel/UCZjaIYhRs2wAZn9tPED_yuw}{youtube})
        \item \href{http://kompendium.meetit.pl/kurs}{komendium.meetit.pl}
        \item \href{https://cp-algorithms.com/}{cp-algorithms.com} (artykuły po angielsku)
        \item książka "Wprowadzenie do algorytmów" by Cormen, podobno w bibliotece szkolnej
        \item książka "Zaprzyjaźnij się z algorytmami" - podstawy
        \item książki Olimpiady Informatycznej
        \begin{itemize}
            \item Przygody Bajtazara
            \item W Poszukiwaniu Wyzwań
            \item niebieskie książeczki - omówienia zadań z OI
        \end{itemize}
    \end{itemize}
\end{itemize}

\end{document}

